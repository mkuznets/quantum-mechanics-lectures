\chapter{Движение в центрально-симметричном поле}

\section{Центрально-симметричное поле. Гамильтониан частицы в сферических координатах. Разделение переменных в центрально-симметричном поле.}

\begin{defn}
Если поле центрально-симметрично, то $U(\vr) \equiv U(r)$
\end{defn}

Произведём переход к координатам $(r, \theta, \phi)$.

Из \eqref{eq:8_4_6}:
\begin{equation}
\label{eq:9_1_1}
\op{H} = \frac{\op{\vp}^2}{2m} + U(r) = - \frac{-\hbar^2}{2m} \brs{\frac{1}{r^2} \pd{}{r} \brc{r^2 \pd{}{r}}} - \frac{\op{\vec{l}}^2}{r^2} + U(r)
\end{equation}

Интегралы движения (из упр. 5 задания 2 и упр. 6 задания 1):
$$
\brs{\op{H}, \op{l}_\alpha} = 0 ~~~~ \brs{\op{H}, \op{\vec{l}}^2} = 0
$$

Из \eqref{eq:8_4_5}:
$$
\brs{\op{\vec{l}}, \op{\vec{l}}_\alpha} = 0
$$

\begin{equation}
\label{eq:9_1_2}
\bk{\vec{r}}{nlm} \equiv \psi_{nlm}(\vr) = R_{nl}(r) \cdot Y_{lm}(\theta, \phi)
\end{equation}
где $n$ -- главное, $l$ -- орбитальное, а $m$ -- магнитное квантовые числа.

\section{Уравнение для радиальной функции}

$$
-\frac{\hbar^2}{2m} \brs{ \frac{1}{r^2} \pd{}{r}\brc{r^2 \pd{}{r}} - \frac{\op{\vec{l}}^2}{r^2}} \psi(r, \theta, \phi) + U(r)\psi(r, \theta, \phi) = E \psi(r, \theta, \phi)
$$

Будем искать решение в виде:
$$
\begin{dcases}
\left. \psi(r, \theta, \phi) \right|_{\text{\eqref{eq:9_1_2}}} = R_{nl}(r) Y_{lm}(\theta, \phi) \\
\op{\vec{l}}^2 Y_{lm}(\theta, \phi) = l (l+1) Y_{lm}(\theta, \phi)
\end{dcases}
$$

Подставляем их в уравнение:
$$
-\frac{\hbar^2}{2m} \brs{ \frac{1}{r^2} \D{}{r} \brc{r^2 \D{}{r} R_{nl}(r) } - \frac{l(l+1)}{r^2} R_{nl}(r) } + U(r) R_{nl}(r) = E R_{nl}(r)
$$

Уравнение для радиальной волновой функции:
\begin{equation}
\label{eq:9_2_1}
\boxed {
	\frac{1}{r^2} \pd{}{r} \brc{r^2 \D{}{r} R_{nl}(r)} - \frac{l(l+1)}{r^2} R_{nl}(r) + \frac{2m}{\hbar^2} \brc{E - U(r)} R_{nl}(r) = 0
}
\end{equation}

$$
\int \abs{\psi_{nlm}(\vr)}^2 d\vr = \iiint \abs{R_{nl}(r)}^2 \cdot \abs{Y_{lm}(\theta, \phi)}^2 r^2 dr d\Omega = \int_{0}^{\infty} \abs{R_{nl}(r)}^2 r^2 dr \underbrace{\oint \abs{Y_{lm}(\theta, \phi)}^2 d\Omega}_{=1 \text{ из \eqref{eq:8_5_4}}} = 1
$$

Отсюда получаем условие нормировки для радиальной части волновой функции:
\begin{equation}
\label{eq:9_2_2}
\boxed {
	\int_{0}^{\infty} \abs{R_{nl}(r)}^2 r^2 dr = 1
}
\end{equation}