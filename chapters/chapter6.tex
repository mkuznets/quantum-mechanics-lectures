\chapter{Теория представлений}

\section{Матричные представление}

В случае когда спектр оператора является дискретным, возникают матричные представления. Пусть спектр оператора $\op{G}$ (например, оператора Гамильтона, хотя с равным успехом это может быть оператор другой физической величины) является чисто дискретным, так что

$$
\op{G} \ket{n} = g_n \ket{n},~~ \ket{n} \equiv \ket{\psi_n}
$$%
%
где $n = 1, 2,...$ -- индекс состояний дискретного спектра. Из условия полноты \eqref{eq:3_3_11}

$$
\sum_n \ket{n}\bra{n} = \mathds{1}
$$%
%
следует, что произвольный вектор состояния $\ket{\psi}$ может быть разложен по собственным векторам $\ket{n}$ следующим образом
$$
\ket{\psi}=\sum_n \ket{n}\bk{n}{\psi}
$$%
%
Как видимо, набор всех амплитуд $\bk{n}{\psi}$ и волновая функция $\psi$ в данном $g$-представлении (если $\op{G}=\op{H}$ -- оператор Гамильтона, то говорят об энергетическом $E$-представлении) обладают одинаковой информативностью, поэтому в $g$-представлении волновую функцию можно изобразить в виде бесконечной <<столбцовой матрицы>>:
$$
\bk{n}{\psi} = \begin{pmatrix}
\bk{1}{\psi}\\
\bk{2}{\psi}\\
...
\end{pmatrix}
$$

Результат действия произвольного оператора $\op{F}$ в $g$-представлении на вектора состояния $\ket{\psi}$ имеет вид
\begin{equation}
\begin{split}
\label{eq:6_1_1}
\bk{n}{\op{F}\psi} = \bkh{n}{\op{F}\brc{\sum_{n'} \ket{n'}\bra{n'}} \psi} = \sum_{n'} \bfk{n}{\op{F}}{n'}\bk{n'}{\psi} = \\ = \begin{pmatrix}
F_{11} & F_{12} & \cdots\\
F_{21} & F_{22} & \cdots\\
\cdots & \cdots & \cdots
\end{pmatrix} \begin{pmatrix}
\bk{1}{\psi}\\
\bk{2}{\psi}\\
...
\end{pmatrix}
\end{split}
\end{equation}%
%
Следовательно, матрица \eqref{eq:6_1_1} с элементами $F_{nn'} = \bfk{n}{\op{F}}{n'}$ есть оператор $\op{F}$ в $g$-представлении. В таком случае говорят о матричном представлении оператора $\op{F}$ в базисе $\{\ket{n}\}$ собственных векторов оператора $\op{G}$. Величина $\bk{n}{\op{F}\psi}$ есть результат умножения матрицы $F_{nn'}$ на столбец $\bk{n'}{\psi}$.

В матричном представлении оператор $F_{nn'}$ является эрмитово сопряжённым по отношению к оператору $\op{F}$, если
$$
\left .\bfk{n'}{\op{F}^\dag}{n}\right|_{\text{\eqref{eq:3_2_6}}} = \bfk{n}{\op{F}}{n'}^*
$$%
%
Иными словами, элементы эрмитово сопряжённой матрицы $F^\dag$ получаются из элементов матрицы $F$ с помощью операций транспонирования ($n' \rightleftharpoons n $) и комплексного сопряжения.

Для эрмитовой матрицы $F$, когда $\op{F}^\dag = \op{F}$,
$$
F_{n'n}=F_{nn'}^*
$$%
%
т.е. эрмитовому оператору соответствует эрмитовая матрица. Для диагональных матричных элементов $n' = n$: $F_{nn} = F^*_{nn}$, т.е. они для эрмитовой матрицы действительны.

В матричном представлении произведение двух произвольных операторов $\op{A}$ и $\op{B}$ сводится к произведению матриц, соответствующих этим операторам:%
$$
\bfk{n}{\op{A}\op{B}}{n''} =
	\sum_{n'} \bfk{n}{\op{A}}{n'} \bfk{n'}{\op{B}}{n''} =
	\sum_{n'} A_{nn'} B_{n'n''}
$$

В матричном представлении задача на собственные векторы и собственные значения оператора $\op{F}$ принимает вид%
$$
\op{F}\ket{f}=f\ket{f} \xrightarrow{(6.1.1)} \sum_{n'} F_{nn'} \bk{n'}{f} = f\bk{n}{f}=f\delta_{nn'}\bk{n'}{f}
$$%
%
т.е. сводится к решению системы алгебраических уравнений с нулевой правой частью%
%
$$
\sum_{n'}\brc{F_{nn'}-f\delta_{nn'}}\bk{n'}{f}=0
$$%
%
Эта система разрешима, если детерминант матрицы, составленный из коэффициентов уравнений, обращается в нуль%
%
$$
\det{\norm{F_{nn'}-f\delta_{nn'}}}=0
$$%
%
Корнями этого уравнения являются собственные значения $f$.

В собственном $f$-представлении оператор $\op{F}$ есть диагональная матрица%
%
$$
\bfk{f'}{\op{F}}{f}=f\bk{f'}{f}=f\delta_{f'f}
$$%
%
При этом диагональными элементами являются собственные значения оператора $\op{F}$.


\section{Унитарное преобразование векторов состояний и операторов}

Рассмотрим теперь вопрос, как преобразуются матрицы векторов состояний и операторов при переходе от одного дискретного базиса к другому (или при переходе от одного представления к другому).

Пусть спектры операторов $\op{L}$ и $\op{M}$ дискретны:
$$
\begin{gathered}
\op{L}\ket{\lambda} = \lambda\ket{\lambda},~ \ket{\lambda}=\ket{\chi_\lambda} \\
\op{M}\ket{\mu} = \mu\ket{\mu},~ \ket{\mu}=\ket{\phi_\mu}
\end{gathered}
$$%
%
Вектор состояния $\ket{\psi}$ может быть разложен как по базисным векторам $\ket{\lambda}$, так и по базисным векторам $\ket{\mu}$, т.е.%
%
$$
\ket{\psi}=\sum_\lambda \ket{\lambda}\bk{\lambda}{\psi}=\sum_\mu \ket{\mu}\bk{\mu}{\psi}
$$%
%
Обозначим:%
$$
\begin{gathered}
\bk{\lambda}{\psi} =\ bk{\chi_\lambda}{\psi}= \psi_\lambda \\
\bk{\mu}{\psi}=\bk{\phi_\mu}{\psi}=\psi'_\mu
\end{gathered}
$$%
%
где, в соответствии с предыдущим параграфом, $\psi_\lambda$ -- волновая функция в $\lambda$-представлении (вектор-столбец), $\psi'_\mu$ -- волновая функция в $\mu$-представлении. Тогда каждый элемент $\mu$-базиса можно разложить по элементам $\lambda$-базиса:%
%
\begin{equation}
\label{eq:6_2_1}
\psi'_\mu = \bk{\mu}{\psi} =
	\sum_\lambda \underbrace{\bk{\mu}{\lambda}}_{U_{\mu\lambda}}\bk{\lambda}{\psi} =
	\sum_\lambda U_{\mu\lambda}\psi_\lambda
\end{equation}%
%
где $U_{\mu\lambda}$ 00 матрица перехода от волновой функции $\psi_\lambda$ к волновой функции $\psi'_\mu$. Эта же матрица связывает друг с другом базисные векторы%
%
$$
\begin{gathered}
\bra{\mu}=\sum_\lambda \underbrace{\bk{\mu}{\lambda}}_{U_{\mu\lambda}} \bra{\lambda} = \sum_\lambda U_{\mu\lambda}\bra{\lambda} \\
\ket{\mu}=\sum_\lambda \ket{\lambda} \bk{\lambda}{\mu} = \sum_\lambda U^*_{\mu\lambda}\ket{\lambda} 
\end{gathered}
$$%
%
Из условий ортонормировки имеем:
$$
\begin{gathered}
\bk{\mu}{\mu'}=\delta_{\mu\mu'}\\
\bk{\lambda}{\lambda'}=\delta_{\lambda\lambda'}
\end{gathered}
$$%
%
поэтому
$$
\delta_{\mu\mu'} = \bk{\mu}{\mu'} =
	\sum_\lambda \sum_{\lambda'} \underbrace{\bk{\mu}{\lambda}}_{U_{\mu\lambda}}
		\underbrace{\bk{\lambda}{\lambda'}}_{\delta_{\lambda\lambda'}}
		\underbrace{\bk{\lambda'}{\mu'}}_{\brc{U^\dag}_{\lambda'\mu'}} =
	\sum_\lambda U_{\mu\lambda}\brc{U^\dag}_{\lambda\mu'}
$$%
%
где $\brc{U^\dag}_{\lambda\mu'}$ -- матричные элементы эрмитово сопряжённой матрицы $U^\dag$ (см. предыдущий параграф).

\begin{defn}
Оператор $\op{U}$ называется \underline{унитарным}, если $\op{U}^\dag = \op{U}^{-1}$, т.е. $\op{U}^\dag\op{U}= \op{U}\op{U}^\dag= \mathds{1}$
\end{defn}

Очевидно, что унитарному оператору соответствует унитарная матрица. Действительно, если $\op{U}^\dag\op{U}=\mathds{1}$, то%
%
$$
\sum_{n'}U_{nn'} \brc{U^\dag}_{n'n''}=\delta_{nn''}
$$%
%
отсюда $U^\dag=U^{-1}$.

Итак, мы доказали, что преобразование вектора состояния из одного представления в другое является унитарным:
\begin{equation}
\label{eq:6_2_2}
\ket{\psi'}=\op{U}\ket{\psi}
\end{equation}%
%
Здесь, согласно соотношению \eqref{eq:6_2_1}, унитарный оператор $\op{U}$ переводит вектор некоторого состояния $\ket{\psi}$ в исходном $\lambda$-представлении в вектор этого же состояния в другом $\mu$-представлении.

Пусть $\op{F}$ -- оператор физической величины в исходном $\lambda$-представлении, $\op{F'}$ -- соответствующий оператор в новом $\mu$-представлении. Выразим их друг через друга. Для этого запишем преобразование матричных элементов при переходе от одного представления к другому:%
%
\begin{equation}
\begin{gathered}
\label{eq:6_2_3}
F_{\mu\mu'} =  \bfk{\mu}{\op{F}}{\mu'} = \left.
	\sum_\lambda \sum_{\lambda'} 
		\underbrace{\bk{\mu}{\lambda}}_{U_{\mu\lambda}}
		\underbrace{\bfk{\lambda}{\op{F}}{\lambda'}}_{F_{\lambda\lambda'}}
		\bk{\lambda'}{\mu'} \right|_{\text{\eqref{eq:6_2_1}}} = \\ =
	\sum_\lambda \sum_{\lambda'} U_{\mu\lambda} F_{\lambda\lambda'} \brc{U^\dag}_{\lambda'\mu'}
\end{gathered}	
\end{equation}%
%
Следовательно матрицы $F'$ и $F$ одного и того же оператора $\op{F}$ в $\mu$- и $\lambda$-представлениях связаны друг с другом следующим образом%
\begin{equation}
\label{eq:6_2_4}
\boxed{F'=U F U^\dag}
\end{equation}%
%
Умножая обе части равенства \eqref{eq:6_2_4} слева на $U^\dag$, а справа на $U$, получаем%
\begin{equation}
\label{eq:6_2_5}
\boxed{F=U^\dag F' U}
\end{equation}%
%
Из матричных равенства \eqref{eq:6_2_4} и \eqref{eq:6_2_5} следуют унитарные преобразования оператора $\op{F}$ физической величины между исходным и новым представлениями:%
%
\begin{equation}
\label{eq:6_2_4_add}
\boxed{\op{F'}=\op{U} \op{F} \op{U}^\dag}
\tag{\ref{eq:6_2_4}$'$}
\end{equation}%
%
\begin{equation}
\label{eq:6_2_5_add}
\boxed{\op{F}=\op{U}^\dag \op{F'} \op{U}}
\tag{\ref{eq:6_2_5}$'$}
\end{equation}%

В классе унитарных преобразований векторов состояний и операторов справедливы следующие утверждения:

\begin{enumerate}

\item Скалярное произведение любых двух векторов $\ket{\psi_1}$ и $\ket{\psi_2}$ инвариантно к их унитарному преобразованию:
$$
\bk{\psi'_1}{\psi'_2}=\bk{\psi_1}{\psi_2}
$$

\item Унитарные преобразования не меняют собственных значений наблюдаемой (эрмитова оператора): если $\op{F}\ket{f}=f\ket{f}$, то $\boxed{\op{F'}\ket{f'}=f\ket{f'}}$

\item Унитарные преобразования не нарушают эрмитовости оператора: если $\op{F}^\dag=\op{F}$, то $(\op{F'})^\dag=\op{F'}$

\item Унитарные преобразования сохраняют коммутационные соотношения: \\
если $[\op{F}, \op{G}]=\op{K}$, то $[\op{F'}, \op{G'}]=\op{K'}$

\item Значения матричных элементов и средние значения наблюдаемых не меняются при унитарных преобразованиях: $\boxed{\bfk{\psi_1}{\op{F}}{\psi_2}=\bfk{\psi'_1}{\op{F'}}{\psi'_2}}$
\end{enumerate}

\begin{excr}
Доказать утверждения 1-5.
\end{excr}

Из сформулированных выше утверждений можно сделать вывод, что все физические (наблюдаемые) величины, которые можно сравнить с данными эксперимента, не меняются при унитарном преобразовании (в частности, при смене представления). Поэтому при унитарности преобразования физическое содержание теории остаётся неизменным. В классической механике преобразования с такими свойствами называются каноническими (см.~\llref{45}{1}). Таким образом, унитарные преобразования в квантовой механике -- аналог канонических преобразований в классической.

В заключение этого раздела вернёмся к Теореме~\ref{theorema_iv_chapter} (\sref{1}{4}) об одновременной измеримости двух физических величин $\op{F}$ и $\op{G}$. Докажем достаточность утверждения теоремы при вырожденном спектре оператора $\op{F}$ и имеет собственные значения $f_n \rightarrow \brcr{\ket{\psi_n^{(i)}}} \equiv \ket{n^{(i)}},~i = 1..k$ где $k$ -- кратность вырождения собственного значения $f_n$. Но тогда любая линейная комбинация векторов $\ket{n^{(1)}}, \ket{n^{(2)}}, ..., \ket{n^{(k)}}$ также является собственным вектором оператора $\op{F}$, отвечающим собственному значению $f_n$.

Пусть унитарная матрица $U$ есть матрица перехода от одного ортонормированного набора векторов $\brcr{\ket{n^{(i)}}}$ к другому $\brcr{\ket{n'^{(i)}}}$, т.е.%
%
$$
\ket{n'^{(i)}}=\sum_j U_{ij} \ket{n^{(j)}}
$$%
%
Тогда в представлении векторов $\brcr{\ket{n'^{(i)}}}$ матрица оператора $\op{G}$ имеет вид%
$$
G'=U G U^\dag
$$%
%
Из линейной алгебры известно, что подходящим унитарным преобразованием любая эрмитовая матрица может быть переведена к диагональному виду. После такого приведения имеем:
$$
G'_{ij}=g_i \delta_{ij}~~~\text{или}~~~\op{G}\ket{n'^{(i)}}=g_i\ket{n'^{(i)}}
$$%
%
Таким образом векторы $\ket{n'^{(1)}}, \ket{n'^{(2)}}... \ket{n'^{(k)}}$ являются искомыми собственными векторами как для оператора $\op{F}$, так и для оператора $\op{G}$. Теорема доказана.

\section{Координатное и импульсное представления}

Состояние частицы (квантовой системы) в точке $\vr$ по определению задаётся вектором состояния $\ket{\vr}$, состояние частицы с импульсом $\vp$ -- вектором $\ket{\vp}$. Поскольку координата -- физическая величина, то её соответствует оператор $\op{\vr}$, для которого векторы $\ket{\vr}$ -- собственные векторы с соответствующими собственными значениями:
\begin{equation}
\label{eq:6_3_1}
\op{\vr}\ket{\vsr}=\vsr\ket{\vsr}
\end{equation}%
%
Здесь $\vsr$ -- собственное значение оператора координаты (действительный радиус-вектор), и оно соответствует тому, что частица находится в точке с координатами $\vsr$, а $\vsr$ -- собственный вектор, отвечающий собственному значению $\vsr$.

Те же самые слова можно произнести и для импульса частицы:
\begin{equation}
\label{eq:6_3_2}
\op{\vp}\ket{\vsp}=\vsp\ket{\vsp}
\end{equation}%
%
Здесь $\vsp$ -- собственное значение оператора импульса, и оно соответствует тому, что частица обладает импульсом $\vsp$.

Домножая соотношение \eqref{eq:6_3_2} слева на $\bra{\vec{p''}}$, получим
$$
\bfkh{\vec{p''}}{\op{\vp}}{\vsp}=\vsp\bkh{\vec{p''}}{\vsp}
$$%
%
и учитывая, что собственные векторы состояний непрерывного спектра нормированы на $\delta$-функцию (см.~\eqref{eq:3_4_3}):%
$$
\bkh{\vec{p''}}{\vsp} = \delta(\vec{p''}-\vsp)
$$%
%
находим, что в $\vp$-представлении <<матрица>> оператора импульса имеет вид:%
%
\begin{equation}
\label{eq:6_3_3}
\boxed{\bfkh{\vec{p''}}{\op{\vp}}{\vsp}=\vsp \delta(\vec{p''}-\vsp)}
\end{equation}

Аналогично с учётом непрерывного характера спектра оператора координаты \eqref{eq:6_3_1} (его спектр -- всё вещественное трёхмерное пространство) находим, что <<матрица>> оператора координаты в собственном, т.е. координатном $\vr$-представлении, имеет вид:%
\begin{equation}
\label{eq:6_3_4}
\boxed{\bfkh{\vec{r''}}{\op{\vr}}{\vsr}=\vsr \delta(\vec{r''}-\vsr)}
\end{equation}%
%
Из \sref{3}{3}, оператор
\begin{equation}
\label{eq:6_3_5}
\op{P}_{\vr}=\ket{\vr}\bra{\vr}
\end{equation}%
%
проектирует любой вектор $\ket{\psi}$ на базисный вектор состояний с координатой $\vr$
\begin{equation}
\label{eq:6_3_6}
\op{P}_{\vr} \ket{\psi} = \ket{\vr}\bkh{\vr}{\psi} = \bkh{\vr}{\psi} \ket{\vr}
\end{equation}%
%
Здесь проекция $\bkh{\vr}{\psi}$ показывает, как выглядит состояние $\ket{\psi}$ в точке $\vr$. Но это не что иное как, по определению, волновая функция:
\begin{equation}
\label{eq:6_3_7}
\boxed{
	\bkh{\vr}{\psi} \equiv \psi(\vr)
}
\end{equation}%
%
Разложим $\forall \ket{\psi} \in \mathcal{H}$ по базису $\brcr{\ket{\vr}}$ с учётом условия полноты для векторов состояний непрерывного спектра (\sref{4}{3}):
\begin{equation}
\label{eq:6_3_8}
\ket{\psi}=\int \ket{\vr}\bkh{\vr}{\psi} \, d\vr
\end{equation}%
%
Анализируя соотношения \eqref{eq:6_3_7} и \eqref{eq:6_3_8}, можно заключить, что континуум компонент вектора $\ket{\psi}$ в базисе $\brcr{\ket{\vr}}$ есть комплекснозначная функция вещественного аргумента $\vr$ или волновая функция в координатном ($\vr$-) представлении.

Аналогично можно построить волновую функцию в импульсном ($\vp$-) представлении
\begin{equation}
\label{eq:6_3_8_add}
\ket{\Phi}=\int \ket{\vp}\bkh{\vp}{\Phi} \, d\vp
\tag{\ref{eq:6_3_8}$'$}
\end{equation}%
где%
\begin{equation}
\label{eq:6_3_7_add}
\boxed{\bkh{\vp}{\Phi} \equiv \Phi\brc{\vp}}
\tag{\ref{eq:6_3_7}$'$}
\end{equation}%
%
следует рассматривать как комплекснозначную функцию вещественной переменной $\vp$. Её физический смысл $\abs{\Phi\brc{\vp}}^2 d\vp$ -- вероятность обнаружить значения импульса частицы в интервале $(\vp, \vp+d\vp)$ был выявлен ранее в \sref{1}{2}, где $C(\vp) \equiv \Phi(\vp)$.

Пусть $\ket{\psi}\equiv\ket{\vp}$, тогда \eqref{eq:6_3_6} имеем:
\begin{equation}
\label{eq:6_3_6_add}
\op{P}_{\vr}\ket{\vp}=\bk{\vr}{\vp}\ket{\vr}
\tag{\ref{eq:6_3_6}$'$}
\end{equation}%
%
где волновая функция $\bk{\vr}{\vp}$ описывает состояние частицы с определённым импульсом $\vp$ в точке с радиусом-вектором $\vr$, т.е. свободную частицу, которой отвечает волна де Бройля:%
%
\begin{equation}
\label{eq:6_3_9}
\boxed{
	\left. \Psi_{\vp}(\vr,t) \right|_{\text{\eqref{eq:2_1_2}}} = 
	\left. \frac{1}{(2\pi \hbar)^{(3/2)}} e^{\frac{i}{h} (\vp\vr-Et)}  \right|_{\text{\eqref{eq:6_3_6_add}}} = 
	\bk{\vr}{\vp} e^{-iEt/\hbar}
}
\end{equation}%
%
Теперь легко получить связь между волновыми функциями в координатном и импульсном представлениях:%
%
\begin{equation}
\label{eq:6_3_10}
\psi(\vr) \equiv \bk{\vr}{\psi} = 
\int \underbrace{  \bk{\vr}{\vp}  }_ {\psi_{\vp}(\vr)}  \underbrace{ \bk{\vp}{\psi} }_{\psi(\vp)} \, d\vp =
\int \psi_{\vp}(\vr) \psi(\vp) \, d\vp
\end{equation}%
-- аналог построения волнового пакета из волн де Бройля (см.~\eqref{eq:2_1_4}). Также можно записать:%
%
\begin{equation}
\label{eq:6_3_11}
\psi(\vp) \equiv \bk{\vp}{\psi} = 
\int \underbrace{  \bk{\vp}{\vr}  }_ {\bk{\vr}{\vp}^*} \bk{\vr}{\psi} \, d\vr =
\int \psi^*_{\vp}(\vr) \psi(\vr) \, d\vr
\end{equation}%
-- Фурье-образ волновой функции $\psi{\vr}$ (см.~\eqref{eq:2_1_5}). 

Определим, как действуют операторы координаты $\vr$ и импульса $\vp$ на произвольный вектор состояний в собственном базисе (представлении). Подействуем сначала на произвольный вектор состояния оператором координаты $\vr$:%
%
\begin{equation}
\label{eq:6_3_12}
\op{\vr} \ket{\psi} = \ket{\phi}
\end{equation}%
%
где $\ket{\phi}$ -- неизвестный пока вектор. В базисе собственных состояний оператора координаты вид неизвестного состояния получается разложением его по базису состояний $\brcr{\ket{\vr}}$.Проекции этого разложения по определению дают значения состояния $\ket{\phi}$ в точке с координатой $\vr$, т.е. волновую функцию $\phi(\vr)$:%
%
\begin{equation}
\label{eq:6_3_13}
\begin{gathered}
	\left.\bk{\vr}{\phi} \equiv \phi(\vr)\right|_{\text{\eqref{eq:6_3_12}}} =
	\bfkh{\vr}{\op{\vr}}{\psi} = 
	\bfkh{\vr}{\op{\vr} \cdot \mathds{1}_{\vsr}}{\psi} = \\ =
	\bfkh{\vr}{\op{\vr}\int d\vsr}{\vr'} \bk{\vsr}{\psi} =
	\left.\int d\vsr \bfkh{\vr}{\op{\vr}}{\vsr} \psi(\vsr) \right|_{\text{\eqref{eq:6_3_4}}} = \\ =
 	\int d\vsr \, \vsr \delta(\vr - \vsr) \psi(\vsr) = \vr \psi(\vr) = \phi(\vr)
\end{gathered}
\end{equation}%
%
Как видим, в координатном представлении действие оператора координаты $\op{\vr}$ на волновую функцию $\psi{\vr}$ сводится к умножению последней не действительный вектор $\vr$.%
%
\begin{excr}
Следуя схеме \eqref{eq:6_3_13}, показать, что действие произвольной функции от оператора координаты $U(\op{\vr}) \equiv \op{U}(\vr)$ на волновую функцию $\psi(\vr)$ также сводится к её умножению на вещественную функцию $U(\vr)$, т.е. что $U(\op{\vr})=U(\vr)$.
\end{excr}%
%
В \sref{2}{2} при определении оператора физической величины через среднее значение уже были получины в координатном представлении упомянутые выражения для оператором $\op{\vr}$ и $U(\op{\vr})$ (см.~\eqref{eq:2_2_4} и \eqref{eq:2_2_5} соответственно).

Подействуем теперь оператором импульса $\op{\vp}$ на произвольный вектор:
\begin{equation}
\label{eq:6_3_14}
\op{\vp} \ket{\psi} = \ket{\chi}
\end{equation}%
%
В базисе собственных состояний $\brcr{\ket{\vp}}$ вид неизвестного состояния получается разложением его по данному базису. Проекции этого разложения по определению дают значения состояния $\ket{\chi}$ в точке с импульсом $\vp$:%
%
\begin{equation}
\label{eq:6_3_15}
\begin{gathered}
	\left.\bk{\vp}{\chi} \equiv \chi(\vp)\right|_{\text{\eqref{eq:6_3_14}}} =
	\bfkh{\vp}{\op{\vp}}{\psi} = 
	\bfkh{\vp}{\op{\vp} \cdot \mathds{1}_{\vsp}}{\psi} = \\ =
	\bfkh{\vp}{\op{\vp}\int d\vsp}{\vp'} \bk{\vsp}{\psi} = 
	\left.\int d\vsp \bfkh{\vp}{\op{\vp}}{\vsp} \psi(\vsr) \right|_{\text{\eqref{eq:6_3_3}}} = \\ =
	\int d\vsp \, \vsp \delta(\vp - \vsp) \psi(\vsp) = \vp \psi(\vp) = \chi(\vp)
\end{gathered}
\end{equation}%
%
Таким образом получаем, что, как и для оператора координаты в координатном представлении, действие оператора импульса в собственном представлении сводится к умножению волновой функции в импульсном представлении на значение импульса $\vp$.

\begin{excr}
\label{excr2}
Следуя схеме \eqref{eq:6_3_15}, показать, что действие произвольной функции от оператора импульса $F(\op{\vp}) \equiv \op{F}(\vp)$ на волновую функцию $\psi(\vp)$ сводится её умножению на вещественную функцию $F(\vp)$, т.е. $F(\op{\vp})=F(\vp)$.
\end{excr}

Теперь займёмся важной задачей преобразования уравнения Шрёдингера из координатного представления в импульсное. Стационарное уравнение Шрёдингера для частицы с массой $m$ в координатном представлении имеет вид:
\begin{equation}
\label{eq:6_3_16}
\op{H}\ket{\psi} = \brc{\frac{\op{\vp}^2}{2m} + \op{U}(\vr)}\ket{\psi} = E\ket{\psi}
\end{equation}%
%
Проводим преобразования \eqref{eq:6_3_16} по известной схеме:
$$
\begin{gathered}
\bfk{\vp}{\op{H}}{\psi} = \bfk{\vp}{\op{H} \cdot \mathds{1}_{\vsp}}{\psi} =
\int d\vsp \, \bfk{\vp}{\op{H}}{\vsp} \bk{\vsp}{\psi} = \\ =
\int d\vsp \, \left[ \bfkh{\vp}{\op{T}}{\vsp} + \bfkh{\vp}{\op{U}(\vr)}{\vsp} \right] \psi(\vsp)
\end{gathered}
$$%
%
В соответствии с Упр.~\ref{excr2} для действия $F(\op{\vp})$ получаем:%
%
\begin{equation}
\begin{split}
\label{eq:6_3_17}
	\left. \int d\vsp \, \bfkh{\vp}{\op{T}}{\vsp} \psi(\vsp) \right|_{\text{упр.~\ref{excr2}}} =
	\int d\vsp \, \bfkh{\vp}{ \frac{\vsp^2}{2m} }{\vsp} \psi(\vsp) = \\ =
	\int d\vsp \, \frac{\vsp^2}{2m} \underbrace{ \bkh{\vp}{\vsp} }_{\delta(\vp - \vsp)} \psi(\vsp) = \frac{\vp^2}{2m} \psi(\vp)
\end{split}
\end{equation}%
%
Матричный элемент $\bfkh{\vp}{\op{U}(\vr)}{\vsp}$ нам пока неизвестен. <<Расщепим>> его единичными операторами:%
%
$$
\begin{gathered}
	\bfkh{\vp}{\op{U}(\vr)}{\vsp} = \bfkh{\vp}{ \mathds{1}_\vr \cdot \op{U(\vr)} \cdot \mathds{1}_\vsr}{\vsp} =
	\iint d\vr d\vsr \, \underbrace{ \bk{\vp}{\vr} }_{\bk{\vr}{\vp}^*} \underbrace{ \bfkh{\vr}{\op{U}(\vr)}{\vsr} }_{U(\vsr)\delta(\vr - \vsr)} \bkh{\vsr}{\vsp}
\end{gathered}
$$
%
После одного интегрирования по координате $\vsr$, получаем:

\begin{equation}
\begin{gathered}
\label{eq:6_3_18}
\bfkh{\vp}{\op{U}(\vr)}{\vsp} =
	\left. \int d\vr \, \psi^*_{\vp}(\vr) U(\vr) \psi_{\vsp}(\vr) \right|_{\text{\eqref{eq:6_3_9}}} = \\ =
	\boxed{
		\frac{1}{(2\pi\hbar)^3} \int d\vr e^{-\frac{i}{\hbar} \brc{\vp - \vsp} \vr } U(\vr) = W(\vp - \vsp)
	}
\end{gathered}
\end{equation}%
%
Итак, матричный элемент оператора потенциальной энергии в импульсном представлении связан с Фурье-образом потенциала $U(\vr)$. В $\vr$-представлении оператор потенциальной энергии $U(\vr)$ является локальным (он зависит от одной переменной), а в $\vp$-представлении оператор потенциальной энергии $W(\vp - \vsp)$ становится нелокальным, т.е. зависит от двух переменных $\vp$ и $\vsp$.

С учётом преобразований \eqref{eq:6_3_17} и \eqref{eq:6_3_18}, стационарное уравнение Шрёдингера в импульсном представлении принимает вид:
\begin{equation}
\label{eq:6_3_19}
	\boxed{\frac{p^2}{2m} \psi(\vp) + \int W(\vp - \vsp) \psi(\vsp) \, d\vp  = E\psi(\vp)}
\end{equation}%
%
Следовательно, в $\vp$-представлении уравнение Шрёдингера в общем случае становится интегральным. Разумеется, интегральное уравнение \eqref{eq:6_3_19} получается только при условии, что Фурье-образ потенциала \eqref{eq:6_3_18} существует.

Уравнение Шрёдингера в импульсном представлении, конечно, эквивалентно уравнению Шрёдингера в координатном, поскольку одно получается из другого фактически преобразованием Фурье. Однако, в некоторых случаях уравнение в $p$-представлении решается проще, и этим можно воспользоваться в практических расчётах (см. например задачу 1 из второго задания). В рамках разработанных в этом параграфе схем преобразований можно также получить выражения для оператора координаты в импульсном представлении: $\boxed{\op{\vr} = i\hbar \pd{}{\vp}}$, и для оператора импульса в координатном представлении: $\boxed{\op{\vp} = -i\hbar \pd{}{\vr}}$.

\section{Оператор эволюции. Представление Шрёдингера и Гайзенберга. Уравнение Гайзенберга для операторов физических величин}

Представления Шрёдингера и Гайзенберга -- это не совсем те представления, о которых говорилось в предыдущих параграфах. Здесь речь пойдёт не о выборе базиса для волновой функции и не о представлении векторов и операторов в виде матриц. В контексте этого параграфа мы будем говорить о представлении изменения состояния во времени или о разных способах описания {\em временной эволюции} квантовой системы. Фактически речь пойдёт о разных способах введения времени в формальную схему квантовой механики. В классической механике обычно применяется термин <<уравнения движения>>, ибо рассматривается изменение пространственного положения частицы. Мы будем применять термин <<уравнение эволюции во времени>>. Он является более общим, т.е. могут изменяться и внутренние свойства системы, не сводящиеся к перемещению в пространстве. Иногда вместо представлений говорят о {\em картине} Шрёдингера или Гайзенберга.

В {\em представлении Шрёдингера} операторы, как правило, не зависят явно от времени, а вся зависимость от времени входит через векторы состояния $\ket{\Psi(t)}$. Зависимость вектора состояния от времени определяется уравнением Шрёдингера

\begin{equation}
\label{eq:6_4_1}
	i\hbar \pd{}{t} \ket{\Psi(t)} = \op{H}\ket{\Psi(t)}
\end{equation}

Пусть по определению

\begin{equation}
\label{eq:6_4_2}
	\ket{\Psi(t)} = \op{U}(t)\ket{\Psi(0)}
\end{equation}%
%
где $\op{U}(t)$ -- оператор эволюции, отображающий пространство векторов $\ket{\Psi(0)}$ в начальный момент времени $t=0$ в пространство векторов $\ket{\Psi(t)}$ в момент времени $t$. Из сохранения во времени нормировки состояний

$$
\bk{\Psi(t)}{\Psi(t)} = \bfk{\Psi(0)}{\op{U}^\dag(t) \op{U}(t) }{\Psi(0)} = \bk{\Psi(0)}{\Psi(0)}
$$%
%
следует, что
$$\op{U}^\dag(t) \op{U}(t) = \mathds{1}$$
т.е. унитарность оператора эволюции (см.~\sref{2}{6}).

Для получения в явном виде оператора $\op{U}(t)$ подставим определение \eqref{eq:6_4_2} в уравнение Шрёдингера \eqref{eq:6_4_1}:

$$
\brcr{i\hbar \pd{}{t} \op{U}(t) - \op{H}\op{U}(t)} \ket{\Psi(0)} = 0 
$$

Поскольку состоания $\ket{\Psi(0)}$ задано, т.е. произвольное, то соотношения \eqref{eq:6_4_1} и \eqref{eq:6_4_2} сводятся к операторному уравнению

\begin{equation}
\label{eq:6_4_3}
	i\hbar \pd{}{t} \op{U}(t) = \op{H}\op{U}(t)
\end{equation}%
%
Если $\op{H}$ не зависит от $t$, т.е. $\partial \op{H}/\partial t = 0$, то решение операторного уравнения \eqref{eq:6_4_3} с начальным условием $\op{U}(0) \ne \mathds{1}$ (см.~\eqref{eq:6_4_2}) можно записать в виде

\begin{equation}
\label{eq:6_4_4}
	\boxed{\op{U}(t)=e^{-\frac{i}{\hbar} \op{H}t}}
\end{equation}%
%
\begin{defn}
Операторная экспонента задаётся рядом степенных операторов
\begin{equation}
\label{eq:6_4_5}
e^{-\frac{i}{\hbar} \op{H}t} \equiv \sum_{n=0}^{\infty} \frac{1}{n!} \brc{-\frac{i}{\hbar} \op{H}t}^n
\end{equation}
\end{defn}

Мы видим, что изменение с течением времени волновой функции может быть представлено как результат унитарного преобразования, оператор которого $\op{U}$ зависит от времени.

\begin{defn}
Описание временной эволюции квантовой системы, когда вектор состояния (или волновая функция) зависит от времени, а операторы не зависят, называется \underline{представлением Шрёдингера}
\end{defn}

Рассмотрим другой способ описания временной эволюции системы или другой способ введения времени -- {\em представление Гайзенберга}. Идея состоит в том, чтобы перенести зависимость от времени на операторы, тогда как волновая функция от времени не зависит, т.е. она неподвижна. Иными словами, попытаемся для описания временной эволюции системы использовать только начальное состояние $\ket{\Psi(0)}$. Далее совершим над векторами состояния в представлении Шрёдингера унитарное преобразование посредством зависящего от времени оператора $\op{U}(t)$. Пусть по определению $\ket{\psi_H} \equiv \ket{\psi_S(0)}$, тогда

\begin{equation}
\label{eq:6_4_6}
	\left. \ket{\psi_H} \equiv \ket{\psi_S(0)} \right|_\text{\eqref{eq:6_4_2}} \equiv 
	\left.  \op{U}^\dag(t) \ket{\psi_S(t)} \right|_\text{\eqref{eq:6_4_4}} =
	e^{\frac{i}{\hbar}\op{H}_St} \ket{\psi_S(t)}
\end{equation}

\begin{excr}
Доказать, что если $\op{U}(t)=\exp\brc{-\frac{i}{\hbar}\op{H}t}$, то $\op{U}^\dag(t) = \exp\brc{\frac{i}{\hbar}\op{H}t}$, используя определение \eqref{eq:6_4_5}.
\end{excr}

Согласно \eqref{eq:6_2_2} и \eqref{eq:6_2_4_add} имеем соответствующее преобразования для операторов:

\begin{equation}
\label{eq:6_4_7}
\boxed{
	\op{F_H}(t) = \op{U}^\dag(t) \op{F_S} \op{U}(t) = e^{(i/\hbar) \op{H_S}t} \op{F_S} e^{-(i/\hbar) \op{H_S}t}
}
\end{equation}

\begin{excr}
Доказать операторное равенство:
$$
e^{\xi \op{A}} \op{B} e^{-\xi \op{A}} = \op{B} + \xi \brs{\op{A}, \op{B}} + \frac{\xi^2}{2!} \brs{\op{A}, \brs{\op{A}, \op{B}}} + ...
$$
где $\xi$ -- произвольный комплексный параметр, в нашем случае: $\xi = \frac{i}{\hbar} t$, $\op{A} = \op{H_S}$, $\op{B} = \op{F_S}$
\end{excr}

Применим равенство из упражнения к правой части \eqref{eq:6_4_7}:

\begin{equation}
\label{eq:6_4_8}
\op{F_H}(t) = \op{F_S} + \frac{i}{\hbar} t \brs{\op{H_S}, \op{F_S}} - \frac{t^2}{2\hbar^2} \brs{\op{H_S}, \brs{\op{H_S}, \op{F_S}}} + ...
\end{equation}%
%
Из \eqref{eq:6_4_8} следует, что если $\op{F_S}$ является оператором интеграла движения, т.е. $\brs{\op{H_S}, \op{F_S}} = 0$ (см.~\sref{2}{5}), то $\op{F_H}(t) = \op{F_S}$, т.е. гайзенберговские операторы интегралов движения не зависят от времени и совпадают с соответствующими операторами в представлении Шрёдингера. В частности, это относится к гамильтониану системы

$$
\op{H_H} = \op{H_S} = \op{H}
$$%
%
так что вместо \eqref{eq:6_4_7} теперь может написать

\begin{equation}
\label{eq:6_4_9}
\op{F_H}(t) = e^{(i/\hbar)\op{H}t} \op{F_S} e^{-(i/\hbar)\op{H}t}
\end{equation}%
%
Продифференцировав \eqref{eq:6_4_9} по времени (разумеется, считая при этом операторы $\op{F_S}$ и $\op{H}$ не содержащими $t$), получим

\begin{equation}
\label{eq:6_4_10}
\boxed{
	\D{}{t}\op{F_H}(t) = \frac{i}{\hbar} \brs{\op{H}, \op{F_H}}
}
\end{equation}%
%
-- уравнение движения для гайзерберговского оператора $\op{F_H}(t)$.

\begin{defn}
Представление, в котором эволюция во времени переносится на операторы, а векторы состояния от времени не зависят, называется \underline{представлением Гайзерберга}.
\end{defn}

Уравнение \eqref{eq:6_4_10} очень похоже на соотношение \eqref{eq:5_2_6}. Однако заметим, что последнее представляет собой определение оператора скорости изменения физической величины в картине Шрёдингера, тогда как соотношение \eqref{eq:6_4_10} записано для гайзерберговского оператора.

Описание эволюции системы в представлении Гайзенберга физически совершенно эквивалентно описанию в представлении Шрёдингера, т.е. эти представления связаны унитарным преобразованием (унитарно эквивалентны). Однако конкретные вычисления для определённой задачи в одном представлении могут оказаться значительно проще, чем в другом.

Отметим, что в представлении Гайзенберга так же, как в и представлении Шрёдингера, остаётся полная свобода выбора обобщённых координат системы. В обоих случаях существуют координатное, импульсное и множество других представлений в том смысле, что в качестве независимых переменных волновых функций могут быть выбраны $\vr$, $\vp$ и другие физические величины.

% ??? Было на лекции, в конспекте нет
% Из \eqref{eq:5_2_6}:
% $$
% \D{\op{F}}{t} = \pd{\op{F}}{t} + \frac{i}{\hbar} \brs{\op{H}, \op{F}}
% $$
