\chapter{Совместная измеримость физических величин}

\section{Условия одновременной измеримости физических величин. Коммутаторы.}

Пусть $\widehat{F}, \widehat{G}$ -- наблюдаемые и имеют дискретные спектры.
$$
\begin{array}{lcl}
\widehat{F}, \widehat{G}:  & F \rightarrow \widehat{F} \rightarrow \{f_n\}, \; n=0,1,2,\dots\\
                           & G \rightarrow \widehat{G} \rightarrow \{g_m\}, \; m=0,1,2,\dots
\end{array}
$$

$\ket{\phi}$ --- общий собственный вектор операторов $\widehat{F}$ и $\widehat{G}$.

\begin{equation}
\label{eq:4_1_1}
\begin{cases}
\widehat{F} |\phi \rangle = f_n |\phi \rangle\\
\widehat{G} |\phi \rangle = g_m |\phi \rangle
\end{cases} 
\end{equation}

Обозначим $F = f_n$, $G=g_m$, тогда

$$\ket{\phi} \equiv \ket{f_n g_m} \equiv \ket{nm}$$

$\{\ket{nm}\}$ --- полная система собственных векторов.

Тогда:

\begin{equation}
\label{eq:4_1_2}
\forall \ket{\psi} \in \mathcal{H} ~\rightarrow ~ \ket{\psi} = \underbrace{\sum_{n,m}\ket{nm}\bra{nm}}_{=\mathds{1}} \psi \rangle 
\end{equation}

$$P_{\ket{\psi}}(F = f_n, G = g_m) = \abs{\bk{nm}{\psi}}^2$$

\begin{defn}
Физические величины $F$ и $G$ одновременно (совместно) измеримы, если их операторы $\widehat{F}$ и $\widehat{G}$ обладают общей полной системой собственных векторов (собственных функций).
\end{defn}

\begin{thm}\label{theorema_iv_chapter}
Для того, чтобы физические велчины $F$ и $G$ были совместно измеримы необходимо и достаточно, чтобы операторы $\widehat{F}$ и $\widehat{G}$ коммутировали, то есть: 
$$[\widehat{F}, \widehat{G}] \equiv \widehat{F}\widehat{G} - \widehat{G}\widehat{F} = 0$$
\end{thm}

\begin{proof}
1) Необходимость. Пусть $F$ и $G$ совместно измеримы, тогда:

$$ \{\ket{nm}\} \in \mathcal{H},~~ \forall \ket{\psi} \in \mathcal{H} ~\rightarrow ~$$
$$(\widehat{F}\widehat{G} - \widehat{G}\widehat{F}) \ket{\psi} = \sum_{n,m} (\widehat{F}\widehat{G} - \widehat{G}\widehat{F}) \ket{nm}\bk{nm}{\psi} = \sum_{n,m} (f_n g_m - g_m f_n) \ket{nm}\bk{nm}{\psi} = 0$$
Что и требовалось доказать

2) Достаточность. 

а) Пусть $[\widehat{F}, \widehat{G}] = 0$ и $\widehat{F}$ имеет невырожденный спектр.

\begin{equation}
\label{eq:4_1_3}
f_n \rightarrow \ket{n}:~~ \widehat{F} \ket{n} = f_n \ket{n}
\end{equation}

$$\widehat{G}(\widehat{F} \ket{n}) = \left. \widehat{F}(\widehat{G}\ket{n}) \right|_{(4.1.3)} = f_n (\widehat{G}\ket{n})$$

$\widehat{G} \ket{n}$ --- собственный вектор оператора $\widehat{F}$ с собственным значением $f_n$

Собственные векторы $\widehat{G} \ket{n}$ и $\ket{n}$ соответствуют одному собственному значению, а значит коллинеарны: $\widehat{G} \ket{n} = g_m \ket{n}$

$$\ket{n} \equiv \ket{f_n g_m} \equiv \ket{nm}$$ (полная система)

б) Случай вырожденного спектра $\widehat{F}$ доказан в гл.6

Таким образом, если $[\widehat{F},\widehat{G}]\ne0$, то $\widehat{F}$ и $\widehat{G}$ могут иметь общий собственный вектор. Но из теоремы следует, что из малого числа собственных векторов нельзя построить полную систему.
\end{proof}

\begin{defn}
Число $f$, входящее в собственные значения оператора физической величины $\widehat{F}$, называют квантовым числом, характеризующим состояние системы $\ket{\psi_f}\equiv\ket{f}$.
\end{defn}

Набор взаимно коммутирующих операторов $(\widehat{G}_1, \widehat{G}_2...)$ коммутирующих с $\widehat{F}$ даёт полный набор квантовых чисел в случае вырожденного спектра.\footnote{Хм...}

$\psi_{f g_1 g_2 ...}(\vec{q})$ или собственные векторы $\ket{f g_1 g_2 ...}$ дают полное описание квантового состояния системы.

\begin{defn}
Набор взаимно коммутирующих операторов, собственные значения (квантовые числа) которых однозначно определяют квантовое состояние системы, называют \underbar{полным набором совместных наблюдаемых}.
\end{defn}

\section{Соотношение неопределённостей}

Пусть $\widehat{F}$ и $\widehat{G}$ --- операторы физических величин $F$ и $G$, то есть $\widehat{F}^+=\widehat{F}$, $\widehat{G}^+=\widehat{G}$ и $[\widehat{F},\widehat{G}]=i\widehat{K}$.

\begin{thm}
В произвольном квантовом состоянии выполняется соотношение неопределённостей:
$$\boxed{\avgh{\brc{\widehat{F}-\avg{\widehat{F}}}^2} \avgh{\brc{\widehat{G}-\avg{\widehat{G}}}^2} \geqslant \frac{\avg{\widehat{K}}^2}{4}}$$
\end{thm}

\begin{proof}
1) Если $[\widehat{F},\widehat{G}]=i\widehat{K}$, то $\widehat{K}^+=\widehat{K}$, т.к.
\begin{equation}
\label{eq:4_2_1}
[\widehat{F},\widehat{G}]^+=-i\widehat{K}^+
\end{equation}

С другой стороны:
\begin{equation}
\label{eq:4_2_2}
[\widehat{F},\widehat{G}]^+=
(\widehat{F}\widehat{G}-\widehat{G}\widehat{F})^+=\widehat{G}^+\widehat{F}^+-\widehat{F}^+\widehat{G}^+=
-[\widehat{F}^+,\widehat{G}^+]=
-[\widehat{F},\widehat{G}]=
-i\widehat{K}
\end{equation}

Сравнивая правые части \eqref{eq:4_2_1} и \eqref{eq:4_2_2}, получаем $\widehat{K}^+=\widehat{K}$

2) В состоянии $\ket{\psi}$:~ $\avg{F}=\bfk{\psi}{\widehat{F}}{\psi}$, то есть:
$$\left.
\begin{gathered}
\Delta\widehat{F}=\widehat{F}-\avg{\widehat{F}}\cdot\mathds{1}\\
\Delta\widehat{G}=\widehat{G}-\avg{\widehat{G}}\cdot\mathds{1}
\end{gathered}
\right\}~\text{операторы отклоления от среднего}
$$

$$[\Delta\widehat{F},\Delta\widehat{G}]=i\widehat{K}$$

3) $\ket{\phi}=(\Delta\widehat{F}-i\gamma\Delta\widehat{G})\ket{\psi}$, где $\gamma$ --- вещественный параметр.

Проведём сопряжение:
$$\bra{\phi}=\bra{\psi}(\Delta\widehat{F}-i\gamma\Delta\widehat{G})^+=\bra{\psi}(\Delta\widehat{F}+i\gamma\Delta\widehat{G})$$

4) $\bk{\phi}{\phi}=\norm{\ket{\phi}}=\bra{\psi}(\Delta\widehat{F}+i\gamma\Delta\widehat{G})(\Delta\widehat{F}-i\gamma\Delta\widehat{G})\ket{\psi}=\bfk{\psi}{\Delta\widehat{F}^2}{\psi}+\gamma^2\bfk{\psi}{\Delta\widehat{G}^2}{\psi}+\gamma\bfk{\psi}{\widehat{K}}{\psi} \geqslant 0$

Значит, у этого уравнения относительно $\gamma$ не более одного корня, т.е. можно записать условие на его дискриминант: 

$$\avg{\widehat{K}}^2-4\avg{\Delta\widehat{F}^2}\avg{\Delta\widehat{G}^2} \leqslant 0$$

Общее соотношение неопределённостей:
$$\boxed{\avg{\Delta\widehat{F}^2}\avg{\Delta\widehat{G}^2}\geqslant\frac{\avg{\widehat{K}}^2}{4}}$$

Дисперсия: $$\avgh{\Delta\widehat{F}^2}=\avgh{(\widehat{F}-\avg{\widehat{F}})^2}=\avgh{\widehat{F}^2}-\avgh{\widehat{F}}^2$$ 
\end{proof}

\begin{exmpl} Выведем соотношение неопределённостей для координаты и импульса:

$$\widehat{F}=\widehat{x}=x$$
$$\widehat{G}=\widehat{p}_x=-i\hbar\frac{\partial}{\partial{x}}$$

$$[\widehat{x},\widehat{p}_x]\psi(x)=x(-i\hbar\frac{\partial}{\partial{x}})\psi(x)-(-i\hbar\frac{\partial}{\partial{x}})x\psi(x)=i\hbar\psi(x)~\to~\widehat{K}=\hbar$$


$$\boxed{\avgh{(\Delta\widehat{x})^2}\avgh{(\Delta\widehat{p}_x)^2} \geqslant \frac{\hbar^2}{4}}$$
\end{exmpl}