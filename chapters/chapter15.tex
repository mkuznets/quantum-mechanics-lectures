\chapter{Сложение моментов}

\section{Сложение моментов}

Пусть есть два момента $j_1$ и $j_2$, а пространство представимо в виде: $\mathcal{H} = \mathcal{H}^{(1)} \otimes \mathcal{H}^{(2)}$.

$[\op{j}_{1 \alpha}, \op{j}_{2 \beta}] = 0$ при $\alpha, \beta = x, y, z$, так как моменты действуют в разных подпространствах.

Как мы уже знаем, 
\begin{gather*}
\op{\vec{j}}_1^2\ket{j_1 m_1} = j_1(j_1 + 1)\ket{j_1 m_1}  \\
\op{j}_{1z} \ket{j_1 m_1} = m_1\ket{j_1 m_1}, 
\end{gather*}
где $m_1 = -j_1, -j_1 + 1, ... , j_1 - 1, j_1$.

Аналогично
\begin{gather*}
\op{\vec{j}}_2^2\ket{j_2 m_2} = j_2(j_2 + 1)\ket{j_2 m_2}  \\
\op{j}_{2z} \ket{j_2 m_2} = m_2\ket{j_2 m_2}, 
\end{gather*}
где $m_2 = -j_2, -j_2 + 1, ... , j_2 - 1, j_2$.

Размерность пространства $\mathcal{H} = \mathcal{H}^{(1)} \otimes \mathcal{H}^{(2)}$ равна $(2j_1 + 1)(2j_2 + 1)$.

Введем оператор полного углового момента для систем 1 и 2:
$$
\op{\vec{j}} = \op{\vec{j}}_1 + \op{\vec{j}}_2
$$

Перейдем от базиса $\ket{j_1 m_1 j_2 m_2}$ к базису $\ket{j m j_1 j_2}$. 

\begin{excr}
Доказать, что это можно сделать, т.е. что $\op{\vec{j}^2}$, $\op{j_z}$, $\op{\vec{j}}_1^2$, $\op{\vec{j}}_2^2$ взаимно коммутируют.
\end{excr}

Рассмотрим базис $\brcr{\ket{j m j_1 j_2}}$.
\begin{gather*}
\op{\vec{j}}_1^2\ket{j m j_1 j_2} = \lambda_1 \ket{j m j_1 j_2}  \\
\op{\vec{j}}_2^2\ket{j m j_1 j_2} = \lambda_2 \ket{j m j_1 j_2}  \\
\op{\vec{j}}^2\ket{j m j_1 j_2} = j(j+1) \ket{j m j_1 j_2}  \\
\op{j}_{z} \ket{j m j_1 j_2} = m \ket{j m j_1 j_2}, 
\end{gather*}

где $\lambda_1, \lambda_2, j(j+1), m$ требуется найти.\\

Дано: $j_1$, $j_2$, $\ket{j_1 m_1}$, $\ket{j_2 m_2}$. \\

Найти: $j$, $\lambda_1$, $\lambda_2$, $\ket{j m}$ - собственные векторы. \\

$j$ - целое или полуцелое число.

$m = -j, -j + 1, ..., j - 1, j$.

\begin{sloppypar}
\section{Коэффициенты Клебша-Гордана. Полный угловой момент.}
\end{sloppypar}

\begin{equation}
\label{eq:15_2_1}
\boxed{\ket{j m j_1 j_2} = \sum_{m_1, m_2} C^{j m}_{j_1 m_1 j_2 m_2} \ket{j_1 m_1}\ket{j_2 m_2}}
\end{equation}

$C^{j m}_{j_1 m_1 j_2 m_2}$ - коэффициенты Клебша-Гордана.

Из \eqref{eq:15_2_1}:
\begin{gather*}
\lambda_1 = j_1(j_1 + 1) \\
\lambda_2 = j_2(j_2 + 1)
\end{gather*}

1) $\op{j}_z = \op{j}_{1z} + \op{j}_{2z}$.

\begin{equation}
\label{eq:15_2_2}
m\ket{j m} = \sum_{m_1, m_2} C^{j m}_{j_1 m_1 j_2 m_2} (m_1 + m_2)\ket{j_1 m_1}\ket{j_2 m_2}
\end{equation}

Сравнивания \eqref{eq:15_2_1} и \eqref{eq:15_2_2}, получим, что $C^{j m}_{j_1 m_1 j_2 m_2} = 0, m \not = m_1 + m_2$.

$$
\boxed{C^{j m}_{j_1 m_1 j_2 m_2} \not = 0, \text{~если~} m = m_1 + m_2}
$$

\begin{equation}
\label{eq:15_2_3}
\ket{j m} = \sum_{m_1} C^{j m}_{j_1 m_1 j_2 m - m_1} \ket{j_1 m_1}\ket{j_2 m - m_1}
\end{equation}

2) $m_{max} = j_{max} = m_{1max}+m_{2max} = j_1 + j_2$.

Тогда $j = j_{max}$, $m = m_{max} = j_{max}$, $m_1 = j_1$, $m_2 = j_2$.

Из \eqref{eq:15_2_1}:
\begin{gather*}
\ket{j_{max}~j_{max}} = C^{j_1+j_2~j_1 + j_2}_{j_1~j_1~j_2~j_2} \ket{j_1~j_1}\ket{j_2~j_2} \\
\bk{j_{max}~j_{max}}{j_{max}~j_{max}} = 1 \to C^{j_1+j_2~j_1 + j_2}_{j_1~j_1~j_2~j_2} = 1
\end{gather*} 

3) $j_{min}$ - ?

Применим векторную модель сложения моментов (см. \S 31, т. III Л. Л.).

$$
\op{\vec{j}} = \op{\vec{j}}_1 + \op{\vec{j}}_2
$$
$$
j_{min} = \abs{j_1 - j_2}
$$

Пусть $j_1 > j_2$. Тогда $j_{min} = j_1 - j_2$.

$$
\sum_{j = j_1 - j_2}^{j_1 + j_2} (2j + 1) = 2 \sum_{j = j_1 - j_2}^{j_1 + j_2} j + \sum_{j = j_1 - j_2}^{j_1 + j_2} 1 = (2j_1 + 1)(2j_2 + 1)
$$
 - размерность $\mathcal{H}$. 
 
Таким образом, $\ket{j m j_1 j_2}$ - система собственных векторов операторов $\op{\vec{j}}^2$, $\op{j}_z$, $\op{\vec{j}}_1^2$, $\op{\vec{j}}_2^2$. 

\begin{equation}
\label{eq:15_2_4}
\boxed{j = \abs{j_1 - j_2}, \abs{j_1 - j_2} + 1, ..., j_1 + j_2}
\end{equation}

- правило сложения моментов.

$$
m = -j, -j + 1, ..., j -1, j
$$

Остальные коэффициенты Клебша-Гордана можно получить, используя повышающий и понижающий операторы:
\begin{gather*}
\op{j}_- = \op{j}_x - i \op{j}_y = \op{j}_{1x} + \op{j}_{2x} - i \op{j}_{1y} -i\op{j}_{2y} = \op{j}_{1-} + \op{j}_{2-}
\end{gather*}

и учитывая, что $C^{j_1+j_2~j_1 + j_2}_{j_1~j_1~j_2~j_2} = 1$.