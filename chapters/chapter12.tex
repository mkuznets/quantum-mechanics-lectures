\chapter{Стационарная теория возмущений}

В качестве другого примера решения задач квантовой механики приближенными методами выступает {\em теория возмущений} (ТВ). В этом случае гамильтониан может быть представлен в виде
\begin{equation}
\label{eq:12_0_1}
\op{H} = \op{H}^{(0)} + \op{V} =  \op{H}^{(0)}+ \lambda \op{U},
\end{equation}
где $\op{H}^{(0)}$ --- {\em гамильтониан невозмущенной задачи}, $\op{V} = \lambda \op{U}$ --- {\em оператор возмущения}, содержащий малый числовой параметр $\lambda \ll 1$.

Предполагается, что уравнение
$$
\op{H}^{(0)} \ket{\psi^{(0)}} = E^{(0)}\ket{\psi^{(0)}} \text{ --- стационарное УШ}
$$
или
$$
i \hbar \pd{}{t} \ket{\psi^{(0)}} = \op{H}^{(0)} \ket{\psi^{(0)}} \text{ --- нестационарное УШ}
$$
допускает точное решение. Методы отыскания приближенных решений уравнения Шрёдингера с гамильтонианом \eqref{eq:12_0_1} по известным решениям невозмущенной задачи составляют предмет ТВ.

В этой главе мы рассмотрим ТВ для нахождения дискретного спектра гамильтониана $\op{H}$ и соответствующих ему собственных функций (СФ) (собственных векторов (СВ)) {\em стационарной задачи}
\begin{equation}
\label{eq:12_0_2}
\op{H} \ket{\psi_n} = (\op{H}^{(0)} + \lambda \op{U})\ket{\psi_n} = E_n \ket{\psi_n},
\end{equation}
т.~е. \underline{стационарную ТВ}, когда $\op{H}$ не зависит от времени $t$.

\section{Стационарная ТВ в случае невырожденных уровней энергии}

Пусть известны \underline{точные решения} стационарного невозмущенного УШ
\begin{equation}
\label{eq:12_1_1}
\op{H}^{(0)} \ket{\psi^{(0)}_n} = E^{(0)}_n\ket{\psi^{(0)}_n}
\end{equation}

Требуется по этим решения $\ket{\psi^{(0)}_m}$ и $E^{(0)}_m$ построить {\em приближенные решения} $\psi_n$ и $E_n$ уравнения \eqref{eq:12_0_2}. Допустим, что СВ и СЗ (собственные значения) уравнения \eqref{eq:12_0_2} можно представить в виде разложения в ряд по степеням малого параметра $\lambda$ ({\em ряды ТВ}).
\begin{equation}
\label{eq:12_1_2}
\ket{\psi_n} = \sum_{p=0}^\infty \lambda^p \ket{\phi_n^{(p)}} = \ket{\phi_n^{(0)}} + \lambda \ket{\phi_n^{(1)}} + \lambda^2 \ket{\phi_n^{(2)}} + \dots = 
\end{equation}

\begin{equation}
\label{eq:12_1_2_add}
= \ket{\psi_n^{(0)}} + \ket{\psi_n^{(1)}} + \ket{\psi_n^{(2)}} + \dots
\tag{\ref{eq:12_1_2}$'$}
\end{equation}

\begin{equation}
\label{eq:12_1_3}
E_n = \sum_{p=0}^{\infty} \lambda^p \epsilon_n^{(p)} = \epsilon_n^{(0)} + \lambda \epsilon_n^{(1)} + \lambda^2 \epsilon_n^{(2)} + \dots = 
\end{equation}

\begin{equation}
\label{eq:12_1_3_add}
= E_n^{(0)} + E_n^{(1)} + E_n^{(2)} + \dots
\tag{\ref{eq:12_1_3}$'$}
\end{equation}

При $\lambda \to 0$ $E_n \to \epsilon_n^{(0)} \equiv E_n^{(0)}$, а $\ket{\psi_n} \to \ket{\phi_n^{(0)}} \equiv \ket{\psi_n^{(0)}}$.

Такой метод, при котором СВ и СЗ представляются в виде разложения по степеням малого параметра, называется {\em теорией возмущений Релея-Шрёдингера}. Вопросы сходимости рядов здесь не поднимаются, так как рассматривается лишь формальная схема ТВ:

Подставляя \eqref{eq:12_1_2_add} и \eqref{eq:12_1_3_add} в уравнение Шрёдингера \eqref{eq:12_0_2}:
$$
\brc{\op{H}^{(0)} + \op{V}} \brc{\ket{\psi_n^{(0)}} + \ket{\psi_n^{(1)}} + \dots} = \brc{E_n^{(0)} + E_n^{(1)} + \dots} \cdot
$$
$$
\cdot \brc{\ket{\psi_n^{(0)}} + \ket{\psi_n^{(1)}} + \dots},
$$
получаем:

~~~~~Порядок ТВ ~~~~~~~~~~~~~~~~~~~~~~~~~~~~~~~~~~Уравнение:

~(Приближение ТВ)
\begin{eqnarray}
\label{eq:12_1_4} 0 &~~~~~~~~~~~& \brc{\op{H}^{(0)} - E_n^{(0)}} \ket{\psi_n^{(0)}} = 0\\
\label{eq:12_1_5} 1 &~~~~~~~~~~~& \brc{\op{H}^{(0)} - E_n^{(0)}} \ket{\psi_n^{(1)}} = \brc{E_n^{(1)} - \op{V}} \ket{\psi_n^{(0)}} \\
\dots &~&~~~~~~~~~\dots \nonumber \\
 \label{eq:12_1_6} s&~& \brc{\op{H}^{(0)} - E_n^{(0)}} \ket{\psi_n^{(s)}} = \brc{E_n^{(1)} - \op{V}} \ket{\psi_n^{(s-1)}}  + \\
                     &~&+ E_n^{(2)}\ket{\psi_n^{(s-2)}}+\dots+ E_n^{(s)}\ket{\psi_n^{(0)}} \nonumber
\end{eqnarray}

Предположим, что спектр невозмущенной задачи \eqref{eq:12_1_1} - \underline{дискретный} и \underline{невырожденный}, т.~е. $E_n^{(0)} \to \ket{\psi_n^{(0)}}$. Кроме того, считаем, что набор $\brcr{\ket{\psi_n^{(0)}}}$ образует \underline{полную ортонормированную систему векторов}, т.~е. $\bk{\psi_m^{(0)}}{\psi_n^{(0)}} = \delta_{mn}$. Таким образом, переходим к представлению, где $\op{H}^{(0)}$ имеет диагональный вид. Базисом в таком представлении являются СВ невозмущенного гамильтониана $\op{H}^{(0)}$. При этом искомый вектор состояния $\ket{\psi_n}$ можно разложить по полной системе ортонормированных $\brcr{\ket{\psi_m^{(0)}}}$
$$
\ket{\psi_n}= \sum_m c_{nm} \ket{\psi_m^{(0)}}, 
$$
причем $c_{nm} = c_{nm}^{(0)} + c_{nm}^{(1)} + \dots$, где $c_{nm}^{(1)}$ --- того же порядка малости, что и возмущение $\op{V}$.

\subsection{Первое приближение теории стационарных возмущений}

Определим поправки к $n$-ому СЗ и СВ, соответственно чему полагаем $\boxed{c_{nm}^{(0)} =\delta_{mn}}$, т.~е. $c_{nn}^{(0)} = 1,~~~c_{nm}^{(0)} = 0$, если $m \neq n$. Умножим уравнение \eqref{eq:12_1_5} на $\bra{\psi_n^{(0)}}$:

$\bfk{\psi_n^{(0)}}{\op{H}^{(0)} - E_n^{(0)}}{\psi_n^{(1)}} = \bfk{\psi_n^{(0)}}{E_n^{(1)} - \op{V}}{\psi_n^{(0)}}$, т.~е.
$$
\bfk{\psi_n^{(0)}}{ \underbrace{E_n^{(0)} -  E_n^{(0)}}_{=0}}{\psi_n^{(0)}} = E_n^{(1)} - \bfk{\psi_n^{(0)}}{\op{V}}{\psi_n^{(0)}}
$$

Отсюда поправка первого порядка к уровням энергии
$$
\boxed{E_n^{(1)} = \bfk{\psi_n^{(0)}}{\op{V}}{\psi_n^{(0)}} \equiv \bfk{n}{\op{V}}{n} \equiv V_{nn}}
$$
есть среднее значение возмущения в состоянии $\ket{\psi_n^{(0)}}$.

Определение высших поправок к энергии требует вычисления поправок к вектору состояния, поэтому далее получим
\begin{equation}
\label{eq:12_1_7}
\ket{\psi_n^{(1)}} = \sum_m c_{nm}^{(1)} \ket{\psi_m^{(0)}} \equiv \underbrace{\sum_m \ket{\psi_m^{(0)}} \bra{\psi_m^{(0)}}}_{=\op{1}} \ket{\psi_n^{(1)}}
\end{equation}

Подставим \eqref{eq:12_1_7} в \eqref{eq:12_1_5}:
$$
\sum_m c_{nm}^{(1)} \brc{\op{H}^{(0)} - E_n^{(0)}}\ket{\psi_m^{(0)}} = \brc{E_n^{(1)} - \op{V}}\ket{\psi_n^{(0)}}
$$

Умножим только то полученное уравнение на $\bra{\psi_k^{(0)}}$:
$$
\sum_m c_{nm}^{(1)}\brc{E_m^{(0)} - E_n^{(0)}} \bk{\psi_k^{(0)}}{\psi_m^{(0)}} = E_n^{(1)} \bk{\psi_k^{(0)}}{\psi_n^{(0)}} - V_{kn}
$$
или
$$
\sum_m c_{nm}^{(1)}\brc{E_m^{(0)} - E_n^{(0)}} \delta_{km} = E_n^{(1)} \delta_{kn} - V_{kn}
$$

\begin{equation}
\label{eq:12_1_8}
c_{nk}^{(1)} \brc{E_k^{(0)} - E_n^{(0)}} = E_n^{(1)} \delta_{kn} - V_{kn}
\end{equation}

Если $k \neq n$, то из \eqref{eq:12_1_8} следует
\begin{equation}
\label{eq:12_1_9}
c_{nk}^{(1)} = \frac{V_{kn}}{E_n^{(0)} - E_k^{(0)}}
\end{equation}

Если $k = n$, то уравнение \eqref{eq:12_1_8} удовлетворяется тождественно и коэффициент $c_{nn}^{(1)}$ остается произвольным. Его нужно выбрать так, чтобы вектор состояния $\ket{\psi_n} = \ket{\psi_n^{(0)}} + \ket{\psi_n^{(1)}}$ в своей нормировке отличался от 1 лишь на величину второго порядка малости. Для этого надо положить $c_{nn}^{(1)} = 0$ (см. правую часть \eqref{eq:12_1_7}, где $c_{nm}^{(1)} = \bk{\psi_m^{(0)}}{\psi_n^{(1)}}$).

Тогда вектор
\begin{equation}
\label{eq:12_1_10}
\boxed{\ket{\psi_n^{(1)}}= \sum_{k \neq n} \frac{V_{kn}}{E_n^{(0)} - E_k^{(0)}} \ket{\psi_k^{(0)}}}
\end{equation}
будет ортогонален к $\ket{\psi_n^{(0)}}$, а нормировочный интеграл ${\bk{\psi_n}{\psi_n} = \underbrace{\bk{\psi_n^{(0)}}{\psi_n^{(0)}}}_{=1} + \bk{\psi_n^{(1)}}{\psi_n^{(1)}}}$ будет отличен от единицы во втором порядке малости.

\subsection{Энергетическая поправка второго приближения теории стационарных возмущений}

Умножим \eqref{eq:12_1_6} на $\bra{\psi_n^{(0)}}$:
\begin{gather*}
\underbrace{\bfk{\psi_n^{(0)}}{\op{H}^{(0)} - E_n^{(0)}} {\psi_n^{(s)}}}_{\bfk{\psi_n^{(0)}}{E_n^{(0)} - E_n^{(0)}}{\psi_n^{(s)}}} = E_n^{(1)} \bk{\psi_n^{(0)}}{\psi_n^{(s-1)}} - \bfk{\psi_n^{(0)}}{\op{V}}{\psi_n^{(s-1)}} + \\
+ E_n^{(2)}\bk{\psi_n^{(0)}}{\psi_n^{(s-2)}}+\dots+E_n^{(s)}\\
\bfk{\psi_n^{(0)}}{E_n^{(0)} - E_n^{(0)}}{\psi_n^{(s)}} = 0
\end{gather*}

Отсюда следует рекуррентная формула для $E_n^{(s)}$:
$$
\boxed{E_n^{(s)} = \bfk{\psi_n^{(0)}}{\op{V}}{\psi_n^{(s-1)}} - \sum_{t=1}^{s-1} E_n^{(t)} \bk{\psi_n^{(0)}}{\psi_n^{(s-t)}}}
$$

При $s=1$ получаем уже известную формулу для энергетической поправки 1-го порядка.
$$
E_n^{(1)} = \bfk{\psi_n^{(0)}}{\op{V}}{\psi_n^{(0)}}, ~~~~c_{nn}^{(1)} = 0
$$
при $s=2$:
\begin{equation}
\label{eq:12_1_11}
E_n^{(2)} = \bfk{\psi_n^{(0)}}{\op{V}}{\psi_n^{(1)}} - E_n^{(1)}\bk{\psi_n^{(0)}}{\psi_n^{(1)}}
\end{equation}

Подставляя \eqref{eq:12_1_10} в \eqref{eq:12_1_11}, имеем:
\begin{gather*}
E_n^{(2)} = \sum_{k \neq n} \frac{V_{kn}}{E_n^{(0)} - E_k^{(0)}} \underbrace{\bfk{\psi_n^{(0)}}{\op{V}}{\psi_k^{(0)}}}_{V_{nk}}
%\underbrace{\op{H}}_{\text{эрмитов}} = \underbrace{\op{H}^{(0)}}_{\text{эрмитов}} + \underbrace{\op{V}}_{\text{эрмитов}} \to V_{kn} = V_{nk}^{*}
\end{gather*}

В силу эрмитовости гамильтониана и оператора $\op{V}$: $V_{kn} = V_{nk}^{*}$, поэтому

\begin{equation}
\label{eq:12_1_12}
\boxed{E_n^{(2)} = \sum_{k \neq n} \frac{\abs{V_{nk}}^2}{E_n^{(0)} - E_k^{(0)}}} \equiv \sum_{k \neq n} \frac{\abs{\bfk{n}{\op{V}}{k}}^2}{E_n^{(0)} - E_k^{(0)}}
\end{equation}

Как наличие одних уровней влияет на энергетическое положение других?

\begin{figure}[h!]
\begin{minipage}[с]{1cm}
\centering
\begin{tikzpicture}[domain=-1:1]
 \draw[->] (0, -0.5) -- (0, 4) node[right] {$E^\zr_n$};
 \draw[-] (-0.1, 0) -- (0.1, 0) node[right] {$n=0$};
 \draw[-] (-0.1, 0.5) -- (0.1, 0.5) node[right] {$n=1$};
 \draw[-] (-0.1, 2.5) -- (0.1, 2.5) node[right] {$n~(k)$};
 \draw[-] (-0.1, 3) -- (0.1, 3) node[right] {$k~(n)$};
 \node[right] at (0.5, 1.2) {$.$};
 \node[right] at (0.5, 1.5) {$.$};
 \node[right] at (0.5, 1.8) {$.$}; 
\end{tikzpicture}
\end{minipage}
\begin{minipage}[c]{\linewidth-1cm}
\begin{enumerate}
\item Если $k$-ый уровень выше, чем $n$-ый, то ${E_k^{(0)} > E_n^{(0)} \to E_n^{(0)} - E_k^{(0)} < 0}$, т.~е. ${E_n^{(2)} < 0}$. Во втором приближении ТВ верхний уровень <<углубляет>> нижний.

\item Если $k$-ый уровень ниже, чем $n$-ый, то ${E_n^{(0)} > E_k^{(0)} \to E_n^{(0)} - E_k^{(0)} > 0}$, т.~е. ${E_n^{(2)} > 0}$. Во втором приближении ТВ нижний уровень <<выталкивает>> верхний.

Таким образом, как принято говорить, во втором приближении ТВ соседние уровни энергии взаимно <<отталкиваются>>.

\item Если $E_n^{(0)} = E_0^{(0)}$ --- основное состояние, то энергетическая поправка 2-го порядка к основному состоянию всегда отрицательна ($E_n^{(2)} < 0$, так как все члены в сумме \eqref{eq:12_1_12} отрицательны). Иными словами, во втором приближении ТВ основной уровень энергии <<опускается>> вниз.

\end{enumerate}
\end{minipage}
\end{figure}

\subsection{Критерий применимости стационарной ТВ}
 
Очевидно, что в формуле $\ket{\psi_n} = \sum\limits_k c_{nk} \ket{\psi_k^{(0)}}$, где $c_{nk} = c_{nk}^{(0)} + c_{nk}^{(1)} + \dots$, должно быть $\abs{c_{nk}^{(1)}} \ll \abs{c_{nk}^{(0)}} = 1$ или (см. \eqref{eq:12_1_9})

\begin{equation}
\label{eq:12_1_13}
\boxed{\abs{V_{kn}} \ll \abs{E_n^{(0)} - E_k^{(0)}}}
\end{equation}
т.~е. \underline{недиагональные матричные элементы оператора возмущения по модулю должны быть малы по сравнению с абсолютной величиной разности соответствующих энергий невозмущенных уровней}.

Будем называть это условие {\em необходимым условием применимости стационарной ТВ}. Этот критерий \eqref{eq:12_1_13} не работает для близких, а также вырожденных уровней энергии.

\section{Стационарное возмущение вырожденных уровней дискретного спектра. Секулярное уравнение}

Пусть уровень $E_n^{(0)}$ вырожден с кратностью $k$, т.~е. ему соответствует ортонормированный набор СВ $\brcr{\ket{\psi_{n \beta}^{(0)}}}$, где $\beta = 1 \div k$. Обратимся к уравнению \eqref{eq:12_1_4} 0-го порядка ТВ:
\begin{equation}
\label{eq:12_2_1}
\brc{\op{H}^\zr - E_n^\zr} \ket{\psi_{n \beta}^\zr} = 0
\end{equation}

Отсюда следует, что набор состояний $\brcr{\ket{\psi_{n \beta}^{(0)}}}$ \underline{неоднозначен}, так как любая линейная комбинация
$$
\ket{\psi_n^\zr} = \sum_{\beta = 1}^k c_\beta \ket{\psi_{n \beta}^\zr}
$$
тоже удовлетворяет уравнению \eqref{eq:12_2_1}, т.~е.
$$
\brc{\op{H}^\zr - E_n^\zr} \ket{\psi_{n}^\zr} = 0
$$

Обратимся к уравнению \eqref{eq:12_1_5} 1-го порядка ТВ:
$$
\brc{\op{H}^\zr - E_n^\zr} \ket{\psi_{n} ^\one} = \brc{E_n^\one - \op{V}} \ket{\psi_n^\zr}
$$

В дальнейшем из соображений удобства опустим индекс $n$, т.~е.
$$
\ket{\psi^\zr} = \sum_{\beta = 1}^k c_\beta \ket{\psi_{\beta}^\zr}
$$
\begin{equation}
\label{eq:12_2_2}
\brc{\op{H}^\zr - E^\zr} \ket{\psi ^\one} = \brc{E^\one - \op{V}} \ket{\psi^\zr}
\end{equation}

Умножим \eqref{eq:12_2_2} на $\bra{\psi_{\alpha}^\zr}$, где $\alpha \in \beta = 1 \div k$:
$$
\bfk{\psi_{\alpha}^\zr}{\underbrace{\op{H}^\zr - E^\zr}_{=0 \text{ из эрмитовости}}}{\psi^\one} = \bfk{\psi_{\alpha}^\zr}{E^\one - \op{V}}{\psi^\zr},
$$
т.~е.
$$
\sum_{\beta=1}^k c_\beta \bfk{\psi_{\alpha}^\zr}{E^\one - \op{V}}{\psi_{\beta}^\zr} = 0
$$
или
\begin{equation}
\label{eq:12_2_3}
\sum_{\beta=1}^k \brcr{V_{\alpha \beta} - E^\one \delta_{\alpha \beta}} c_\beta = 0
\end{equation}  

Система линейных уравнений \eqref{eq:12_2_3} имеет \underline{нетривиальные решения относительно коэффициентов $c_\beta$}, если 
\begin{equation}
\label{eq:12_2_4}
\boxed{\det \norm{V_{\alpha \beta} - E^\one \delta_{\alpha \beta}} = 0}
\end{equation}

Заметим, что комплексных корней у секулярного уравнения \eqref{eq:12_2_4} нет в силу эрмитовости оператора возмущения $\op{V}$  (см. теорему в конце~\sref{2}{3}, а также~\sref{1}{6}). {\em Секулярное} или {\em вековое уравнение \eqref{eq:12_2_4}} имеет $k$ \underline{действительных} корней, которые и представляют искомые поправки 1-го приближения к энергии уровня $E_n$, т.~е. $k$-кратному СЗ $E_n^{(0)}$ отвечают уровни энергии соответствуют энергии $E_n^\zr + E_{n \mu} ^\one, \mu = 1 \div k$. Если все $E_{n \mu}$ различны, то возмущение \underline{полностью снимает вырождение}. Если некоторые из $E_{n \mu} ^\one$ совпадают, то говорят о \underline{частичном снятии вырождения}.

\subsection{Правильные волновые функции нулевого приближения}

Нумеруя корни секулярного уравнения $E_{n \mu} ^\one$ и подставляя их в \eqref{eq:12_2_3}, найдем конкретные значения коэффициентов
$c_{\mu \beta}$ ($\beta$ напоминает нам о том, что этих коэффициентов $k$ штук для каждого СЗ $E_{n \mu} ^\one$). Тогда СВ в нулевом приближении примут вид
$$
\ket{\tilde{\psi}_{n \mu}^\zr} = \sum_{\beta=1}^k c_{\mu \beta} \ket{\psi_{n \beta}^\zr}, \mu = 1 \div k
$$

Векторы $\ket{\tilde{\psi}_{n \mu}^\zr}$ за счет дополнительных ограничений на коэффициенты $c_{\mu \beta}$ можно сделать ортонормированными ($\bk{\psi_{n \nu}^\zr}{\psi_{n \mu}^\zr} = \delta_{\nu \mu}$) и тогда соответствующие таким состояниям волновые функции называются {\em правильными волновыми функциями нулевого приближения}. Они означают переход к новому базису, в котором матрица оператора возмущения $\op{V}$ в блоке, связанном с вырождением, \underline{имеет диагональный вид}. В старом базисе $\brcr{\ket{\psi_{n \beta} ^\zr}}$ диагонализируется только гамильтониан $\op{H}^\zr$ невозмущенной задачи.

\begin{excr}
Доказать, что в базисе \underline{правильных волновых функций нулевого приближения} вырожденная часть матрицы оператора возмущения $\op{V}$ имеет диагональный вид.
\end{excr}

Диагональный вид вырожденной части матрицы оператора возмущения $\op{V}$ в базисе правильных волновых функций нулевого приближения следует из самой процедуры отыскания корней $E_{n \mu} ^\one$ векового уравнения \eqref{eq:12_2_3}. По своей сути такая процедура означает диагонализацию этой части матрицы оператора возмущения $\op{V}$ или отыскания его собственных значений в виде $E_{n \mu} ^\one$, которым в дальнейшем ставятся в соответствие СВ в виде $\ket{\tilde{\psi}_{n \mu}^\zr}$.

\underline{Происхождение названия.} Если бы эти функции знать заранее (например, угадать), то можно было бы сосчитать положение уровней энергии возмущенной задачи по невырожденной ТВ, ибо недиагональные элементы в новом базисе $\equiv 0$.
 
\section{Квазивырождение, случай двух близких уровней энергии}

Рассмотрим специально случай, когда два уровня энергии оказались близки, так что ТВ для невырожденного спектра неприменима (см. \eqref{eq:12_1_13} --- критерий применимости стационарной ТВ в случае невырожденного спектра). Однако тем не менее спектр гамильтониана $\op{H}^\zr$ невырожден: случай двух близких уровней ({\em квазивырождение}).

Без учета малых взаимодействий
\begin{equation}
\label{eq:12_3_1}
\left \{ 
\begin{matrix}
\op{H}^\zr \ket{\psi_1^\zr} = E_1^\zr \ket{\psi_1^\zr} \\
\op{H}^\zr \ket{\psi_2^\zr} = E_2^\zr \ket{\psi_2^\zr}
\end{matrix}
\right .
\end{equation}
где $\Delta = E_2^\zr - E_1 ^\zr > 0$ (расстояние между невозмущенными уровнями). При учете оператора возмущения $\op{V}$ решение уравнения
\begin{equation}
\label{eq:12_3_2}
\op{H} \ket{\psi} = \brc{\op{H}^\zr + \op{V}} \ket{\psi} = E \ket{\psi}
\end{equation}
следует искать по аналогии со случаем вырожденного спектра в виде линейной комбинации
\begin{equation}
\label{eq:12_3_3}
\ket{\psi} = c_1 \ket{\psi_1^\zr} + c_2 \ket{\psi_2^\zr} 
\end{equation}

Подставляя \eqref{eq:12_3_3} в \eqref{eq:12_3_2}, получим:
$$
\brc{\op{H}^\zr + \op{V}}\brc{c_1 \ket{\psi_1^\zr} + c_2 \ket{\psi_2^\zr}} = E \brc{c_1 \ket{\psi_1^\zr} + c_2 \ket{\psi_2^\zr}} \\
$$

Умножая получившееся уравнение последовательно на $\bra{\psi_1^{(0)}}$ и $\bra{\psi_2^{(0)}}$, получаем систему двух уравнений для нахождения уровней энергии и коэффициентов $c_1,~c_2$ в линейной комбинации \eqref{eq:12_3_3}
\begin{gather*}
  \begin{matrix}
    \begin{pmatrix}
      E_1^\zr + V_{11} - E & V_{12} \\
      V_{21} & E_2^\zr + V_{22} - E
    \end{pmatrix} 
    \begin{pmatrix}
      c_1\\
      c_2
    \end{pmatrix} 
     = 0
  \end{matrix}
\end{gather*}
условием разрешимости которой является \underline{секулярное уравнение}
\begin{equation}
\label{eq:12_3_4}
det \left \| \begin{matrix}
E_1^\zr + V_{11} - E & V_{12} \\ 
\underbrace{V_{21}}_{=V_{12}*} & E_2^\zr + V_{22} - E
\end{matrix}
\right \| = 0
\end{equation}

Раскрывая \eqref{eq:12_3_4} имеем:
$$
E^2 - \brc{E_2^\zr + E_1^\zr + V_{22} + V_{11}}E + \brc{E_1^\zr + V_{11}} \brc{E_2^\zr + V_{22}} - |V_{12}|^2 = 0
$$

(В силу эрмитовости оператора оператора $\op{V}$: $V_{21} = V_{12}^*$, поэтому ${V_{21}V_{12} = V_{12}^* V_{12} = \abs{V_{12}^2}}$). Тогда
\begin{gather*}
E_{2, 1} = \frac{E_2^\zr + E_1^\zr + V_{22} + V_{11}}{2} \pm\\
\pm  \brcr{\frac{1}{4} \brs{ (E_1^\zr + V_{11}) + (E_2^\zr + V_{22}) }^2  - (E_1^\zr + V_{11})(E_2^\zr + V_{22}) + |V_{21}|^2 }^{1/2} = \\
= \frac{1}{2}(E_2^\zr + E_1^\zr + V_{22} + V_{11}) \pm \frac{1}{2} \sqrt{(\underbrace{E_2^\zr -E_1^\zr}_{=\Delta} + V_{22} - V_{11})^2 + 4 |V_{12}|^2}
\end{gather*}

Если $V_{11} = V_{22} = 0$ (что, например, имеет место в случае нечетное функции $V$ (водородоподобный атом в однородном электрическом поле: $V = e \mathcal{E} z$)), то 
$$
E_{2, 1}=\frac{E_2^\zr + E_1^\zr}{2} \pm \frac{1}{2} \sqrt{\Delta^2 + 4 |V_{12}|^2} \to \Delta E = E_2 - E_1 = \sqrt{\Delta^2 + 4 |V_{12}|^2} > \Delta
$$

Отсюда следует, что $\Delta E = E_2 - E_1 = \sqrt{\Delta^2 + 4 |V_{12}|^2} > \Delta$, т.~е. \underline{под действием возмущения близкие уровни энергии <<расталкиваются>>}.

\begin{figure}[h!]
\centering
\begin{tikzpicture}[domain=-5:5]
      \draw [->] (-4,0) node[above] {$E_1^\zr = E_2^\zr$} -- (5,0) node[right] {$\Delta$};
      \draw[->] (0, -4) -- (0,4) node[above] {$E$} ;
      \draw [domain=-4:4, samples=100] plot (\x, {sqrt(\x * \x + 1)}) node[right] {$E_2^\zr, \ket{\psi_2^\zr}$};
      \draw [domain=-4:4, samples=100] plot (\x, {-sqrt(\x * \x + 1)}) node[right] {$E_1^\zr, \ket{\psi_1^\zr}$};	
	\draw[dashed] (-4,-4) -- (4,4);
	\node at (0.8, -2) {$E_1, \ket{\psi_1}$};
    \node at (0.8, 2) {$E_2, \ket{\psi_2}$};	
	\node[below] at (5,4) {$|V_{12}| = 0$};
	\draw[dashed] (-4,4) -- (4,-4);
	\draw (2, 1) -- (-0.1, 1);
	\node[left] at (0.1, 1.6) {$+|V_{12}|$};
	\draw (2, -1) -- (-0.1, -1);
	\node[left] at (0.1, -1.6) {$-|V_{12}|$};
	\draw [<->] (1.9, -1) -- (1.9, 1) node[right] {$2|V_{12}|$}; 
\end{tikzpicture}
\caption{ТВ в случае двух близких по энергии уровней} \label{fig:12_2}
\end{figure}

Из \autoref{fig:12_2} видно, что если $V_{12} \neq 0$, то уровни энергии нигде не пересекаются и минимальное расстояние между ними $2\abs{V_{12}}$ ({\em точка квазипересечения} соответствует точному резонансу уровней в отсутствии возмущения).