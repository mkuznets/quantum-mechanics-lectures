\chapter{Сложный атом}

\section{Вариационный принцип}

Пусть для системы с гамильтонианом $\op{H}$ задача решана, то есть найдены энергии $E_n$ и волновые функции $\Psi_n$, такие что:
$$
\op{H}\Psi_n = E_n \Psi_n,~~ n = 0,1,2...
$$
Волновые функции $\Psi_n$ ортонормированы,
$$
\bk{\Psi_n}{\Psi_m} = \delta_{nm}
$$
и образуют полный базис.\\
$N \geqslant 2$

Рассмотрим произвольную волновую функцию $\Phi$, нормированную на единицу:
$$
\bk{\Phi}{\Phi} = 1
$$
её разложение по базису $\Psi_n$ имеет вид:
$$
\Phi = \sum_n a_n \Psi_n
$$
В силу условий нормировки для коэффициентов $a_n$ имеем:
$$
\sum_n \abs{a_n}^2 = 1
$$
Оценим среднюю энергию $\avg{E}$ системы в состоянии, описываемой произвольной волновой функцией $\Phi$:
\begin{equation}
\label{eq:18_1_1}
\avg{E} \equiv \bfk{\Phi}{\op{H}}{\Phi} \equiv \mathcal{E}[\Phi] = \sum_n \sum_m \bfk{a_n \Psi_n}{\op{H}}{a_m \Psi_m} = \\ = \sum_n \sum_m a_n^* a_m E_m \delta_{nm} = \sum_n \abs{a_n}^2 E_n
\end{equation}
В \eqref{eq:18_1_1} $E_1, E_2, ... \to E_0$,~ $E_n \geqslant E_0$ при $\forall n$

\begin{equation}
\label{eq:18_1_2}
\boxed{\mathcal{E}[\Phi] \geqslant E_0 \sum_n \abs{a_n}^2 = E_0}
\end{equation}
Средние значения, вычисленные с пробными функциями $\Phi$, являются оценками сверху для точной энергии основного состояния.

Если $\Phi = \Psi_0$, то $\mathcal{E}[\Psi_0] = E_0$. (???)

Другая формулировка:
\begin{equation}
\label{eq:18_1_3}
\mathcal{E}[\Phi] \to \min
\end{equation}

\underbar{Метод вариационного исчисления}:
\begin{equation}
\label{eq:18_1_4}
\delta \mathcal{E} = 0 ~~~ \forall \delta\Phi:~ \bk{\Phi}{\Phi} = 1
\end{equation}

\begin{enumerate}
\item Выбирается пробная функция $\Phi = \Phi(q, \alpha_1, \alpha_2 ... )$, зависящая от полного набора координат $q$ и вариационных параметров $\alpha_i$;
\item Вычисляется средняя энергия в состоянии, описываемом данной функцией:
$$
\bfk{\Phi}{\op{H}}{\Phi} = \mathcal{E}(\alpha_1, \alpha_2 ...)
$$
\item Минимизируем $E$ по вариационным параметрам:
$$
\pd{\mathcal{E}}{\alpha_1} = 0,~~ \pd{\mathcal{E}}{\alpha_2} = 0, ... 
$$
Находим соответствующие $\alpha_i^0$.
\end{enumerate}
После этого можно утверждать, что функция $\Phi(q, \alpha_1^0, \alpha_2^0, \alpha_2^0 ...)$ есть наилучшее приближение к $\Psi_0(q)$ в выбранном классе функций. (см. задачи 2 и 3 из

В \eqref{eq:17_2_1} (ШТРИХ!): $\nu = 1s$, $S = 0 \uparrow \downarrow$
$$
\Phi_{g.s}^{sym}(\vr_1, \vr_2) = \phi_{1s}(\vr_1)\phi_{1s}(\vec{r_2}) = \frac{1}{\pi} \brc{\frac{z}{a}}^3 e^{-\frac{z(r_1 + r_2)}{a}}
$$
--- отсутствуют вариционные параметры.

Эффективный заряд $\tilde{Z} < Z = 2$

$$
\bfk{\Phi_{g.s}^{S}}{\op{H}}{\Phi_{g.s}^{S}} = \mathcal{E}(\tilde{Z})
$$
$$
\pd{\mathcal{E}}{\tilde{Z}} = 0 ~~\to~~ \tilde{Z} = Z - \frac{5}{16}
$$
--- задача 3а из второго задания. В задаче 3б два вариационных параметра: $\alpha$ и $\beta$.

\section{Метод Хартри-Фока. Приближения центрального поля. Электронные конфигурации.}
$N$ электронов, заряд $-Ze~~ (e<0)$
\begin{equation}
\label{eq:18_2_1}
\op{H} = \sum_{i=1}^{N} \brc{\frac{\op{\vp_i}^2}{2m} - \frac{Ze^2}{r_i}} + \sum_{i<j} \frac{e^2}{\abs{\vr_i - \vr_j}}
\end{equation}

Эффективное среднее поле --- метод Хартри-Фока (1928-1930).

\begin{enumerate}
\item \begin{equation}
\label{eq:18_2_2}
\mathcal{E}\brs{\{\psi_{\{n_j\}}(\xi_i)\}} = \bfk{\Uppsi^{HF}}{\op{H}}{\Uppsi^{HF}}
\end{equation}

\item $\Uppsi^{HF}(\xi_1, \xi_2, ... \xi_N) = Slater~determinant$ (см. \eqref{eq:16_2_2})

\item 
$$
\mathcal{E}\brs{\{\psi_{\{n_j\}}(\xi_i)\}} = \min~ \forall \delta\psi_{\{n_j\}}(\xi_i)
$$
\end{enumerate}

Система из $N$ уравнений $\to$ уравнения Хартри-Фока. Каждое из них описывает движение выделенного $i$-го электрона в некотором эффективном \underbar{среднем поле} $U_{cc}(\vr_i)$, созданным ядром и остальными электронами атома.
\noindent
$U_{cc}(\vr_i)$ --- центральное, \underbar{самосогласованное} поле (ССП).\\

Из \eqref{eq:18_2_1}:
$$
\op{H} = \underbrace{\sum_{i=1}^{N} \brc{\frac{\op{\vp_i}^2}{2m} + U_{cc}(\vr_i)}}_{\op{H}^{(0)}} + \op{V}
$$
где $\op{H}^{(0)}$ --- центральная, а $\op{V}$ --- нецентральная часть гамильтониана:
$$
\op{V} = \sum_{i<j} \frac{e^2}{\abs{\vr_i - \vr_j}} + \sum_{i=1}^N \brc{-\frac{Ze^2}{r_i} - U_{cc}(\vr_i)}
$$
$$
U_{cc}(\vr_i) =
\begin{cases}
-\frac{Ze^2}{r_i},& r_i \to 0\\
-\frac{(Z-(N-1))e^2}{r_i},& r_i \to \infty\\
\end{cases}
$$

Пусть поле центрально-симметрично: $U_{cc}(\vr_i) = U_{cc}(r_i)$

Энергия зависит от $n$ и $l$ ($E_{nl}$), то есть вырождение по $l$ снимается.
$$

$$